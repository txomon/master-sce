% !TeX spellcheck = en_US
\documentclass[a4paper,conference]{IEEEtran}
\usepackage[T1]{fontenc}
\usepackage[utf8x]{inputenc}


\begin{document}
\title{Hardware module integration in a Zedboard using Linux based embedded system generation toolchains}
\author{Gorka Santos Ortuzar\\<gorkasantosortuzar@gmail.com> \and Javier Domingo Cansino\\<javierdo1@gmail.com>}
\maketitle

\begin{abstract}
The latest trends on hardware design using FPGAs have taken software world into hardware generation.
Hardware systems are no longer thought to have a full customized development but the reuse of 
industry wide-known tools to generate generic system that supports your custom appliance.
This is both needed and possible thanks the lowered prices and high-performance processors embedded in the silicon.
\end{abstract}
\begin{IEEEkeywords}
zynq, xilinx, zedboard, buildroot, linux, fpga
\end{IEEEkeywords}

\IEEEPARstart{C}{reation} of embedded devices have always carried a lot of custom scripts to carry on the build
of a bootable system, as well as the parts needed to build the different tools to accomplish the task. To sum up
plenty of work has been made to create embedded device systems.

Back to the start of the decade, buildroot and openwrt projects among others started to gain relevance, as the
generation of a specific configured system but with generic applications was finally possible. This is new in hardware
world because usually, hardware teams did releases and gave specs to firmware teams, being able to develop customized
systems.


\end{document}